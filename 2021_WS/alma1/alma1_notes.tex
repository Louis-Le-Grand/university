\documentclass[a4paper]{article}
\usepackage[
    lecture,
    course={Alrogithmische Mathematik~I},
    code={V1G5},
    lecturer={Prof.~Dr.\ Jürgen Dölz},
    term={WS~2021/2022}
]{tienuni}

\begin{document}
\maketitle
\tableofcontents

\section*{Einführung}

Gegenstand der algorithmischen Mathematik ist die \emph{Konstruktion} und \emph{Analyse} effizienter Algorithmen (d.\,h.\ auch bestehende Algorithmen effizienter zu gestalten, anstatt Rechenleisung zu erhöhen), um mathematische Problemstellungen mit dem Computer zu lösen. Damit liegt sie im Bereich der angewandten Mathematik.

Konkrete Problemstellungen ergeben sich oft aus technischen Problemen, Naturwissenschaften, Medizin, etc. Ausgehend von dieser Problemstellung können wir das Lösen des Problems oft wie folgt skizzieren:

\begin{center}
    \begin{tikzpicture}
    \end{tikzpicture}
\end{center}

Hierbei ist anzumerken, dass verschiedene Problemstellungen aus den Anwendungen oft zu ähnlichen oder gar gleichen mathematischen Modellen führen. In diesem Fall können wir den abstrakten Lösungsalgorithmus des mathematischen Modells auf alle zugehörigen Anwendungsprobleme anwenden. Ein typisches Beispiel ist das Lösen eines Gleichungssystems.

Im Rahmen dieser Vorlesung konzentrieren wir uns auf die Teilbereiche
\begin{itemize}
    \item Diskrete Mathematik,
    \item Numerik und
    \item Stochastik
\end{itemize}
der angewandten Mathematik.

\section{Zahlendarstelllung am Computer}

\subsection{Zahlensysteme}



\end{document}
